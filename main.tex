\documentclass[11pt]{article}
\usepackage{graphicx}
\usepackage{amsmath}
\usepackage{authblk}
\usepackage[margin=1in]{geometry}
\usepackage{hyperref}

\title{A Universal Law of Adaptation: Logistic Dynamics in Microbial Evolution and Machine Learning}
\author{Jeffrey Cyr}
\affil{Independent Researcher}
\date{}

\begin{document}
	\maketitle
	
	\begin{abstract}
		We demonstrate that microbial evolution and machine learning (ML) training follow a shared logistic adaptation trajectory when expressed in terms of a unified adaptation budget. By fitting both datasets to a logistic growth model and normalizing performance, we show that the curves collapse onto a common form, suggesting a universal law of adaptation across biological and artificial systems.
	\end{abstract}
	
	\section{Introduction}
	Adaptation is a fundamental process in both biological evolution and artificial learning systems. This study explores the hypothesis that microbial evolution and ML training may obey the same underlying adaptation dynamics. By comparing performance trajectories from both domains, we aim to uncover potential universality in how systems improve over time.
	
	\section{Methodology}
	We analyzed two datasets:
	\begin{itemize}
		\item Microbial data: \texttt{adaptation\_metrics\_filled.csv}
		\item ML data: \texttt{ml\_adaptation\_metrics.csv}
	\end{itemize}
	
	Both datasets were fitted to a logistic curve of the form:
	
	
	\[
	P^*(B) = \frac{K}{1 + e^{-r(B - B_0)}}
	\]
	
	
	where \(P^*\) is normalized performance, \(B\) is adaptation budget, \(K\) is plateau performance, \(r\) is adaptation rate, and \(B_0\) is the inflection point.
	
	Performance values were normalized to a common scale and plotted against \(B\).
	
	\subsection*{Data Generation}
	
	The microbial dataset (\texttt{adaptation\_metrics\_filled.csv}) was derived from mutation logs or variant call files. Each row represents a timepoint, with computed fields including variation ($V = N_e \times \mu \times f_b$), default selection ($S = 0.02$), constraint ($C = 0.01$), and normalized performance ($P^*$). If allele frequency was unavailable or constant, mutation count was used; if both were flat, a synthetic progression from 0.1 to 0.9 was applied. The adaptation budget $B$ was computed as:
	
	
	
	\[
	B = \left( \frac{V^\alpha \cdot S^\beta}{C^\gamma} \right) \cdot \text{time}
	\]
	
	
	
	with $\alpha = \beta = \gamma = 1$, and rescaled to match ML budget ranges.
	
	The ML dataset (\texttt{ml\_adaptation\_metrics.csv}) was extracted from training logs of a neural network (e.g., MNIST). Each row corresponds to an epoch, with normalized accuracy as $P^*$ and budget $B$ defined as cumulative training effort: $\text{epoch} \times \text{batch size} \times \text{model size}$.
	
	Both datasets were cleaned to remove NaN and infinite values, and fitted to the logistic model:
	
	
	
	\[
	P^*(B) = \frac{K}{1 + e^{-r(B - B_0)}}
	\]
	
	
	
	
	\section{Results}
	\begin{itemize}
		\item Microbial fit: \(K \approx 6384\), \(r \approx 0.03\), \(B_0 \approx 400\), \(R^2 \approx 0.90\)
		\item ML fit: \(K \approx 0.98\), \(r \approx 21.4\), \(B_0 \approx 0.5\), \(R^2 \approx 0.93\)
	\end{itemize}
	
	Figure~\ref{fig:overlay} shows the normalized overlay plot, demonstrating strong visual alignment between the two systems.
	
	\begin{figure}[h]
		\centering
		\includegraphics[width=0.8\textwidth]{Pstar_vs_B_overlay.png}
		\caption{Normalized performance vs. adaptation budget for microbial and ML systems.}
		\label{fig:overlay}
	\end{figure}
	
	\section{Discussion}
	The similarity in logistic fit parameters and curve shapes between microbial and ML data suggests that adaptation may follow a universal pattern. This has implications for understanding learning and evolution as manifestations of a shared optimization principle. Further research could explore other domains and refine the mathematical framework of adaptation.
	
	\section{Conclusion}
	This experiment provides evidence that microbial evolution and ML training share a common adaptation trajectory. The logistic model fits both datasets well, and the overlay plot supports the universality hypothesis. Future work should expand this framework to other systems and explore theoretical foundations.
	
	
	\section*{Code Availability}
	
	All scripts used to generate the datasets, perform logistic fitting, and produce figures are available upon request. This includes:
	
	\begin{itemize}
		\item Data filling script: \texttt{fill\_adaptation\_parameters\_rescaled.py}
		\item Fitting script with fallback: \texttt{fit\_adaptation\_curves\_safe.py}
		\item Overlay plot script: \texttt{plot\_normalized\_overlay.py}
		\item PDF generation script: \texttt{make\_universal\_adaptation\_report.py}
	\end{itemize}
	


	
	To ensure reproducibility, all code is written in Python and compatible with standard scientific libraries (NumPy, Pandas, SciPy, Matplotlib). The full package can be shared via GitHub or direct archive upon request.
	

	\bibliographystyle{plain}
	\bibliography{bib}
	
	\footnote{Licensed under the MIT License.}
	
	
\end{document}
